% ************************** Thesis Abstract *****************************
% Use `abstract' as an option in the document class to print only the titlepage and the abstract.
\begin{abstract}
Metal halide organic-inorganic hybrid perovskites combine the thermal and mechanical stability of inorganic semiconductors with the structural diversity and processability of organic semiconductors. In particular, 2D lead iodide-based perovskites are self-assembling structures that exhibit strong room temperature exciton effects. Due to high exciton binding energy and oscillator strength, such perovskites are ideal candidates for the production of new mixed light-matter states at room temperature as a result of strong coupling.

Thin films of perovskites with thickness $20-150$\,nm are fabricated via spin coating. Although film morphology depends on the perovskite organic moiety, spin speed and substrate preparation, spinning in a dehydrated atmosphere reliably produces films that are uniform on the micrometre scale over cm$^2$ areas. Ultra-thin perovskite samples are produced using micromechanical exfoliation, and mono- and few-layer areas are identified using optical and atomic force microscopy, with an interlayer spacing of 1.6\,nm. Refractive indices extracted from the optical spectra reveal a sample thickness dependence due to the charge transfer between organic and inorganic layers. These measurements demonstrate a clear difference in the exciton properties between `bulk' (\textgreater15 layers) and very thin (\textless8 layer) regions as a result of the structural rearrangement of organic molecules around the inorganic sheets.

Noble metal island structures can be created using thermal evaporation, and exhibit local surface plasmon resonances in optical spectra. In perovskite-coated gold islands we observe a redshift and broadening of the plasmon resonance as a result of the non-uniform perovskite film. For perovskite-coated silver islands the exciton and plasmon oscillations are more resonant and weakly couple to produce a blueshift in the exciton wavelength of 5\,nm, as well as an increase in the exciton extinction by around 10\%.

A variety of dielectric and metal gratings are used to understand the behaviour of perovskite-coated silver gratings. In these systems we observe evidence for `image-biexcitons'. These composite quasiparticles are formed by the interaction between an exciton and its image in the metal mirror below, with binding energy 100\,meV at room temperature. By changing the polar and azimuthal angles of incident light, we observe strong coupling between excitons and surface plasmon polaritons on the grating, with Rabi splittings of 150 and 125\,meV for the exciton and biexciton respectively.
\end{abstract}
