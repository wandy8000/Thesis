%*****************************************************************************************
%*********************************** Eighth Chapter **************************************
%*****************************************************************************************

\chapter{Conclusions and further work}

\graphicspath{{Chapter8/Figures/}}

Metal-halide based organic-inorganic perovskites are self-assembling semiconductors with many tunable properties. The exact structure formed depends on the stoichiometric mix of organic and inorganic constituents, %WN
and this thesis focused on thin 2D lead iodide perovskites, specifically CHPI (\ce{(C6H9C2H4NH3)2PbI4}) and C$_{12}$PI (\ce{(C12H25NH3)2PbI4}). %end
Such perovskites form layered structures with alternating sheets of inorganic \ce{PbI6} networks and interdigitating organic molecules. Due to a difference in the band gaps of the organic and inorganic constituents, a multiple quantum well structure is formed, and excitons are created and trapped in the inorganic sheets. This reduction in dimensionality (quantum confinement), as well as the reduced refractive index of organic molecules (dielectric confinement) lead to large exciton binding energies. As such the optical properties of such perovskites are dominated by exciton effects even at room temperature. The tunability of the exciton energy via structural changes as well as the many fabrication and processing techniques available make such materials promising candidates for use in optoelectronic devices. We studied the optical properties of perovskite thin films, and hybrid perovskite-metal nanostructures were created to study light-matter coupling between perovskite excitons and surface plasmons.

In Chapter 4 the fabrication of thin CHPI and C$_{12}$PI perovskite films on silica via spin coating was investigated, %WN
and we modified quantum well self-assembly by altering spin coating conditions. Film morphology was strongly affected by the hydrophilicity of the organic moiety, substrate functionalisation and the atmospheric humidity in which spin coating took place. Film thickness was controlled by the spin speed and the initial solution concentration. Thin films with thickness $30-150$\,nm were reliably created, uniform over cm$^2$ areas by spin coating in a dehydrated atmosphere. Substrate functionalisation with aminosilane molecules also improved film uniformity. The results from these experiments were used to create perovskite films for exciton-plasmon coupling experiments in Chapters 6 and 7.%end

In Chapter 5 the optical properties of ultra-thin perovskite samples created using exfoliation were described. We were able to produce monolayer-thick perovskite samples, and found a CHPI layer thickness of 1.6\,nm. Optical spectra were dominated by excitons and the charge transfer between organic and inorganic layers. We differentiated between three main regimes of behaviour. For thickness $>27$\,nm (15 layers) we observed `bulk' thin film behaviour, similar to spin coated films. A structural transition region was found between $15-25$\,nm, where strain and flattening of the inorganic layers led to redshift and broadening of the exciton peak, with corresponding redshift and increase in optical activity of the charge transfer. Finally for the thinnest samples (<15\,nm, 8 layers), relaxation of the inorganic layers produced a blueshift and decrease in linewidth of the exciton peak. %WN
We were unable to directly measure areas $<4$ layers thick due to the small size of such regions ($\sim100$\,nm), so flexible substrates can be used for future exfoliation measurements to reduce fracture and achieve larger monolayers. Alternatively solution-grown crystals can be used, which are around 10 times larger than the solvothermally created microcrystals from our experiments. It has been observed that exciton absorption intensity no longer decreases linearly for films less than 4 monolayers thick \cite{Matsui2002}, however a change in the exciton wavelength was not seen in perovskite films made using layer-by-layer deposition methods (dip coating \cite{Matsui2002} or the Langmuir-Blodgett technique \cite{Era2000}). These experiments demonstrate the sensitivity of the perovskite structure and thus exciton wavelength on experimental conditions, and suggest that organisation of the organic molecules around the inorganic framework is key to controlling material structure at the monolayer scale, and is more likely to be accessible through chemical growth rather than exfoliation. %end

In Chapter 6 the interactions between excitons and localised surface plasmons were explored. Metal island nanostructures were created via the evaporation of noble metals both with and without templating using colloidal monolayers, then coated with perovskite. For Au nanostructures the plasmon was far off-resonance with the exciton, and the non-uniform perovskite coating caused a redshift and broadening of the plasmon resonance. However for Ag nanostructures the two oscillations were more resonant, and we observed weak coupling with $\sim 5$\,nm blueshift in the exciton wavelength, as well as an increase of $\sim 40$\% in the exciton absorption due to field enhancement of the localised surface plasmon. %WN
Although there has been no previous work on coupling between perovskite excitons and localised surface plasmons, absorption or emission enhancement in other excitonic materials is well known, and has been used in light emitting devices and solar cells. For future experiments chemically synthesised nanoparticles should be used as the particle geometry is more strictly controlled, thus the plasmon linewidth can be reduced to better enable strong coupling. %end

In Chapter 7 the coupling between excitons and the modes of perovskite-coated gratings was investigated. Both plasmonic and non-plasmonic gratings were studied, and we were able to identify a range of diffractive, plasmonic and guided modes in polarised optical spectra. The energies and strengths of these modes were sensitive to grating geometry, coating materials and light polarisation, and we were able to create modes resonant with the exciton. We found no modification of the exciton wavefunction except in the case of perovskite-coated Ag gratings. Here we observed the appearance of a second exciton mode, the image biexciton, formed as a result of the interaction between an exciton-polariton and its image in the metallic mirror, and outcoupled via surface plasmon polaritons. The binding energy of the image biexciton was 100\,meV at room temperature, and tunable via a change in the dielectric environment around the exciton. We also observed strong coupling between the exciton-polariton, biexciton, and surface plasmon mode trapped at the bottom of grating slits, with Rabi splittings of 150 and 125\,meV for the exciton and biexciton respectively. %WN
As a result of the plasmonic field enhancement in the perovskite layers, light-matter coupling strength is increased compared to exciton-photon strong coupling. An exciton-plasmon Rabi splitting of 167\,meV has been reported for a perovskite/\ce{SiO2}/planar Ag multilayer system \cite{Symonds2007}, and the Kretschmann configuration was employed to convert far-field light into evanescent waves on the metal surface, which then interacted with excitons. However our grating system is different as light is coupled directly into the layered exciton system and not the plasmonic grating mode. Due to the complex interactions between excitons, plasmons and photons in perovskite-coated gratings, the nature of the observed modes is not yet fully understood, and theory work is ongoing to find a semi-analytical description of the system. Such exciton-plasmon interactions may allow us to control the electron motion and charge carrier behaviour in future optoelectronic devices. %end

%WN
In order to assess the use of 2D perovskites in future devices, both the optical and transport properties of these materials must be understood. This thesis, and many other studies in the literature, have studied the optical properties of perovskites, so future experiments should focus on carrier transport. For example, electrons and holes are clearly able to travel along inorganic sheets, however the structure of electroluminescent devices [Fig.\,\ref{2Fig22}] suggests there may be some charge transfer between the organic and inorganic layers as well, and the exact mechanism by which this occurs has not been clarified. 
%end

Furthermore, the physical stability and lifetime of perovskite samples are concerns when considering their suitability for devices. For example, photo- and humidity-induced degradation are reduced by polymer capping layers, however such structures may have poor transport properties. Moreover, `melting' of organic molecules in commonly used perovskites occur $>100^{\circ}$C [Sec.\,\ref{sec:Cnphases}], thus more thermally stable organic moieties must be found to prevent deterioration. 

More fundamentally, the driving force behind self-assembly of the perovskite structure is not well known. Particularly important is molecular-level understanding of perovskite formation from solution, for example in spin coating. This means the orientation of inorganic layers in coatings over nanostructured surfaces is not currently well known, and this has implications on the optical behaviour of such hybrid systems. Better understanding of the drivers behind the self-assembly process will also allow us to control and tailor perovskite formation to future applications.