%*****************************************************************************************
%*********************************** Eighth Chapter **************************************
%*****************************************************************************************

\chapter{Conclusions and further work}

\graphicspath{{Chapter8/Figures/}}

Metal-halide based organic-inorganic perovskites are self-assembling semiconductors with many flexible and tunable properties. The exact structure formed depends on the stoichiometric mix of organic and inorganic constituents, and this thesis explored the optical behaviour of thin 2D perosvkite samples. Lead iodide perovskites (\ce{(RNH3)2PbI4}) forms a layered structure with alternating inorganic \ce{[PbI6]} octahedral networks and organic molecular sheets. Due to the difference in the band gaps of the organic and inorganic constituents a multiple quantum well structure is formed, so that excitons on created and trapped in the inorganic sheets. This reduction in dimensionality (quantum confinement), as well as the reduced refractive index of organic molecules (dielectric confinement), lead to large exciton binding energy. As such the optical properties of such perovskites are dominated by exciton effects even at room temperature. The tunability of the exciton energy via structural changes or external stimuli, as well as the many fabrication techniques available, make such materials promising candidates for use in future devices.

In Chapter 4 we looked at the fabrication of thin perovskite films by spin coating a solution of the perovskite constituents dissolved in a suitable solvent. The morphology of the thin films were strongly affected by the organic moeity of the perovskite, the substrate functionalisation and the atmosphere in which the spin coating took place. The thickness of the films was controlled by the spin speed and the initial solution concentration. We found we could reliably create thin films with thickness $\sim30-150$\,nm, uniform on $\mu$m lengthscales by spin coating in a dehydrated atmosphere and by functionalising the surface with an aminosilane.

In Chapter 5 we created ultra-thin samples of perovskites using exfoliation, and probed the optical effects of such samples. By changing the substrate to improve optical contrast we were able to view layers just a few monolayers thick microscopically. We were able to produce monolayer-thick perovskite samples, and found a layer thickness in CHPI of 1.6\,nm. The optical spectra were dominated by the exciton and charge transfer between organic and inorganic layers, and we observed three main regimes of behaviour. For areas >27\,nm (15 layers) we observed `bulk' thin film behaviour, similar to spin coated films. A structural transition region was found between $15-25$\,nm, where flattening and straining of the inorganic layers led to a redshift and broadening of the exciton peak, with a corresponding redshift and increase in optical activity of the charge transfer. Finally for the thinnest samples (<15\,nm, 8 layers), relaxation of the inorganic layers produce a blueshift and decrease in linedwidth of the exciton peak.

In Chapter 6 we explored the interaction excitons with localised surface plasmons. Metal island structure were created via the evaporation of noble metals, both with and without templating, then coated with perovskite films. In Au nanostructures, where the plasmon is far off-resonance with the exciton, the perovskite coating caused a redshift and broadening of the localised surface plasmon. However for Ag islands the two oscillators were more in resonance, and thus we observed weak coupling with a $\sim 5$\,nm blueshift of the exciton, as well as an increase in the extinction due to the field enhancement of the localised surface plasmon.

In Chapter 7 we investigated the coupling between excitons and modes on 1D gratings. Both plasmonic and non-plasmonic gratings were created, capable to sustaining a range of diffractive, surface plasmon polariton and localised modes. The energies and strengths of these modes were found to be sensitive to grating geometry, dielectric coatings and light polarisation, and we were able to create modes resonant with the exciton by change the periodicity and incident/azimuthal angle of light. We found no changes to the exciton except in the case of perovskite-coated Ag gratings. In this case we observed the appearance of a second exciton mode due, the image biexciton, formed as a result of the interaction between the exciton and its image in the metallic mirror, outcoupled via the exciton surface plasmon polariton. The binding energy of the image biexciton is 100\,meV, and tunable via a change in the electric environment around the exciton, e.\,g.\,resonance with a surface plasmon polariton. We also observed strong coupling between the exciton, biexciton, and surface plasmon polariton mode on the grating, with Rabi splittings of 150 and 125\,meV for the exciton and biexciton respectively. The out-of-plane image biexciton mediated coupling between perpendicular exciton and surface plasmon modes. Finite element method modelling showed that the surface plasmon mode was trapped in the bottom of the grating slits, and the field enhancement in the perovskite layer as a result enhanced the coupling strength and reduced the cavity length needed when compared to exciton-photon strong coupling. This Rabi splitting in two orders of magnitude larger than what has been observed in inorganic GaAs quantum wells, and is observable even at room temperature.

Although the optical properties of 2D lead iodide perovskites are fairly well understood, fewer studies have focused on studying their transport activities. Although carrier can clearly be transported in the plane of the inorganic sheets, the structure of the electroluminescent devices [Fig.\,\ref{2Fig23}] suggests there may be some charge tranfer between the organic and inorganic layers, the exact nature of which is not well understood. The transport properties of perovskites must be studied more in order to compare them to materials currently used in optoelectronic devices.

More fundamentally, the driving force behind the self-assembly of the perovskite structure is not known. Particularly important is the understanding of the crystal formation on a molecular level from a liquid phase, for example in spin coating. As a result the formation of layers over structured substrates, for example in the grating system, is not known. This layer orientation has great implications in the optical properties of the nanostructures in question. Better understanding would allow us to control and tailor the perovskite formation to suit the needs of the system.