%*****************************************************************************************
%*********************************** Eighth Chapter **************************************
%*****************************************************************************************

\chapter{Conclusions and further work}

\graphicspath{{Chapter8/Figures/}}

Metal-halide based organic-inorganic perovskites are self-assembling semiconductors with many tunable properties. The exact structure formed depends on the stoichiometric mix of organic and inorganic constituents, and in this thesis the optical behaviour of thin 2D lead iodide perovskite samples are explored. Such perovskites form layered structures with alternating sheets of inorganic \ce{PbI6} networks and interdigitating organic molecules. Due to a difference in the band gaps of the organic and inorganic constituents, a multiple quantum well structure is formed, and excitons are created and trapped in the inorganic sheets. This reduction in dimensionality (quantum confinement), as well as the reduced refractive index of organic molecules (dielectric confinement) lead to large exciton binding energy. As such the optical properties of such perovskites are dominated by exciton effects even at room temperature. The tunability of the exciton energy via structural changes as well as the many fabrication and processing techniques available make such materials promising candidates for use in optoelectronic devices.

In Chapter 4 the fabrication of thin perovskite films by spin coating was investigated. Film morphology was strongly affected by the organic moiety of the perovskite, substrate functionalisation and the atmospheric humidity in which spin coating took place. Film thickness was controlled by the spin speed and the initial solution concentration. Thin films with thickness $\sim30-150$\,nm were reliably created, uniform over cm$^2$ areas by spin coating in a dehydrated atmosphere. Substrate functionalisation with aminosilane molecules also improved film quality.

In Chapter 5 the optical properties of ultra-thin perovskite samples created using exfoliation was described. We were able to produce monolayer-thick perovskite samples, and found CHPI layer thickness of 1.6\,nm. Optical spectra were dominated by excitons and the charge transfer between organic and inorganic layers. We differentiated between three main regimes of behaviour. For thickness $>27$\,nm (15 layers) we observed `bulk' thin film behaviour, similar to spin coated films. A structural transition region was found between $15-25$\,nm, where strain and flattening of the inorganic layers led to redshift and broadening of the exciton peak, with corresponding redshift and increase in optical activity of the charge transfer. Finally for the thinnest samples (<15\,nm, 8 layers), relaxation of the inorganic layers produce a blueshift and decrease in linewidth of the exciton peak.

In Chapter 6 the interactions between excitons and localised surface plasmons were explored. Metal island structures were created via the evaporation of noble metals both with and without templating, then coated with perovskite. For Au nanostructures, where the plasmon is far off-resonance with the exciton, the non-uniform perovskite coating caused a redshift and broadening of the plasmon resonance. However for Ag nanostructures the two oscillations were more resonant, and we observed weak coupling with $\sim 5$\,nm blueshift in the exciton wavelength, as well as an increase of $\sim 10$\% in the exciton extinction due to field enhancement of the localised surface plasmon.

In Chapter 7 the coupling between excitons and the modes of perovskite-coated gratings were investigated. Both plasmonic and non-plasmonic gratings were explored, and we were able to identify a range of diffractive, SPP and guided modes in polarised optical spectra. The energies and strengths of these modes were sensitive to grating geometry, coating materials and light polarisation, and we were able to create modes resonant with the exciton. We found no modification of the exciton wavefunction except in the case of perovskite-coated Ag gratings. Here we observed the appearance of a second exciton mode, the image biexciton, formed as a result of the interaction between an exciton and its image in the metallic mirror, and outcoupled via surface plasmon polaritons. The binding energy of the image biexciton was 100\,meV at room temperature, and tunable via a change in the dielectric environment around the exciton. We also observed strong coupling between the exciton, biexciton, and surface plasmon mode trapped at the bottom of grating slits, with Rabi splittings of 150 and 125\,meV for the exciton and biexciton respectively. The plasmonic field enhancement in the perovskite layer increased the coupling strength, and the Rabi splittings are two orders of magnitude larger than what has been observed in inorganic GaAs quantum wells at lower temperatures.

Although the optical properties of 2D lead iodide perovskites are fairly well understood, fewer studies have focused on studying their transport properties. Although carrier transport along the inorganic sheets is clearly possible, the structure of electroluminescent devices [Fig.\,\ref{2Fig22}] suggests there may be some charge transfer between the organic and inorganic layers, the exact nature of which is not well understood. Knowledge of carrier transport in these materials is crucial when assessing their use in optoelectronic devices.

More fundamentally, the driving force behind self-assembly of the perovskite structure is not known. Particularly important is molecular-level understanding of perovskite formation from solution, for example in spin coating. For example, the orientation of inorganic layers in coatings over nanostructured surfaces is not currently well known, and this has implications on the optical behaviour of such hybrid systems. Better understanding of the self-assembly process will also allow us to control and tailor perovskite formation to the required application.