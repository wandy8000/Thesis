%*****************************************************************************************
%*********************************** Fourth Chapter **************************************
%*****************************************************************************************

\chapter{Lead iodide perovskite thin films}

\graphicspath{{Chapter4/Figures/}}

Spin coating is a process that can be used to fabricate thin films on relatively flat substrates. After a solution is deposited, the substrate is accelerated to the desired spin speed and continues rotating to remove excess solution. As the solvent evaporates the material self-assembles to form a solid film. The process is commonly used in industry as it can controllably produce films of thickness 10\,nm to 100\,\textmu m, covering areas with lateral size up to 10\,cm. The film thickness and morphology depend on solution concentration, spin speed and substrate preparation. Although it is possible to create PbI perovskite films using spin coating [Sec.\,\ref{sec:spin}], optimisation is required in order to create continuous and uniform films with thickness under 100\,nm. In this Chapter the formation spin coated films of C$_{12}$PI (\ce{(C12H25NH3)2PbI4}) and CHPI (\ce{(C6H9C2H4NH3)2PbI4}) on silica substrates will be explored.

\section{Spin coating theory}
Although spin coating is experimentally simple, it is complicated to model due to the large number of factors involved. Initially, fluid inertia and surface tension are important as the fluid front spreads out in spiral waves. Solvent evaporation also begins at this point, and a small boundary layer is formed at the liquid-gas interface. At the end of this step a thin and even film forms on the substrate. In the next phase a balance between viscous and centrifugal forces causes fluid flow and thinning. The boundary layer in the solution gradually gets thicker, and the solute concentration varies throughout the film thickness. Viscosity rises as a result of solvent evaporation, eventually inhibiting further flow. Further fluid loss is caused by solvent evaporation, which dominates thinning in the latter stages, and eventually solute concentration becomes uniform throughout the film. There is also a small atmospheric boundary layer above the solution that can influence mass transfer and exert shear forces at the interface \cite{Meyerhofer1978, VanHardeveld1995, Lawrence1988}. The initial acceleration may also affect final film thickness, as too slow an acceleration can lead to complete solvent evaporation before the final spin speed is reached \cite{Birnie2005}.

Various approximations have been used in models of spin coating. Meyerhofer considered the solvent evaporation negligible until the mass loss due to rotational forces fell to the level of the evaporation rate \cite{Meyerhofer1978}, van Hardeveld \textit{et al.\,}used the same principles but modelled the evaporation rate more rigorously in terms of rate of mass transfer at the interface \cite{VanHardeveld1995}, and Lawrence took both the solvent and atmospheric boundary layers into consideration\cite{Lawrence1988}. All three models agree that the final film thickness $h_f$ depends on the angular spin speed $\omega$ as $h_f \propto \omega^{-0.5}$, and this relationship has been experimentally verified \cite{Meyerhofer1978, VanHardeveld1995}. However other exponents have been reported, and Lawrence indicated that an exponent of $\omega$ less than $-0.5$ may be measured in films that complete the full spinning process before reaching $\omega$, and are thus thicker than the calculations anticipate. Shear thinning, where the viscosity of the solution decreases with an increased shear stress, may also be responsible \cite{Lawrence1988}.

\begin{figure}[h!]
\centering    
\includegraphics[width=0.6\textwidth]{Microscope}
\caption{Schematic of optical microscopy and spectroscopy setup, including the reflection and transmission beam paths.}
\label{Microscope}
\end{figure}
\section{Experimental methods}
\label{sec:glass}
Spin coating solutions are prepared by dissolving a chemically synthesised perovskite powder [Sec.\,\ref{sec:solutiongrowth}] in tetrahydrofuran (THF) with a concentration of 20\,mg/ml. Silica substrates are sonicated in a four-step process for approximately 15 minutes per solvent: firstly in a deionised water and detergent solution, then in deionised water, acetone, and finally isopropanol. Three additional substrate preparation techniques are investigated in order to create the most uniform films: \\
(1) \ce{CO2} snowjetting, where a high velocity mix of gaseous and solid carbon dioxide is focused on the substrate, cleaning the surface as a result of momentum transfer and solvent action of the \ce{CO2} \cite{Snowjet}. \\
(2) Silanisation, where substrates are dipped in a 2 vol\% solution of aminopropyltriethoxy silane (APTES) in dry acetone for approximately 90 minutes. A self-assembled monolayer of silane molecules forms on the substrate, and in the case of APTES the surface is functionalised with amine groups. \\
(3) Plasma etching, where substrates are treated using a Diener Electronic Femto plasma system for 5 minutes, using an oxygen plasma to clean contaminants from the substrate. The surface is functionalised with hydroxyl groups and becomes more hydrophilic.

Perovskite films are characterised using optical microscopy and spectroscopy (signals are collected over areas with diameter $\approx 20$\,\textmu m unless otherwise specified) [Fig.\,\ref{Microscope}], and the film thickness is determined by atomic force microscopy (AFM) measurements over scratches in the film.

\begin{figure}[h!]
\centering    
\includegraphics[width=0.88\textwidth]{Fig1}
\caption{BF images at 100$\times$ magnification of spin coated C$_{12}$PI films on silica, with spin speed and substrate preparation as labelled. Note the substrate was heated immediately prior to spin coating for (d). The black marks seen on (g-l) are due to dust particles on the microscope lens.}
\label{4Fig1}
\end{figure}
\section{C$_{12}$PI thin films}
\label{sec:4-1}
Bright field reflection (BF) images of C$_{12}$PI films spin coated on silica at 100$\times$ magnification are shown in Fig.\,\ref{4Fig1}. Due to the hydrophobic nature of the organic molecule, C$_{12}$PI films show significant dewetting without substrate functionalisation [Figs.\,\ref{4Fig1}(a-f)], and for this reason films are not formed on plasma etched substrates (not shown). The non-uniform film in Fig.\,\ref{4Fig1}(d) does not exhibit such dewetting as the substrate was heated before application of the C$_{12}$PI solution, thus the solvent evaporated before excess fluid could be removed. Silanisation increases attractive interactions between constituents of C$_{12}$PI and the substrate, thereby improving film coverage [Figs.\,\ref{4Fig1}(g-i)]. A further snowjet step removes excess APTES that may remain after silanisation, reducing surface roughness and producing the most uniform C$_{12}$PI samples [Figs.\,\ref{4Fig1}(k,l)].

\begin{figure}[h!]
\centering
\includegraphics[width=\textwidth]{Fig2}
\caption{(a) Reflection and (b) transmission spectra for C$_{12}$PI films on silanised silica substrates.}
\label{4Fig2}
\end{figure}
\subsection{Spin speed}
\label{sec:4-2}
Optical spectra of C$_{12}$PI films created on silanised substrates illustrate the general trends observed for all substrate preparations [Fig.\,\ref{4Fig2}]. The exciton appears as a Fano resonance at the expected wavelength of 490\,nm \cite{Pradeesh2009} in reflectivity due to interference between its narrow resonance and the continuum background, while a dip appears in the transmittance spectra due to exciton absorption. Although both phases of C$_{12}$PI are observed for films below 2000\,rpm [Sec.\,\ref{sec:Cnphases}], here we consider only the high energy exciton. Spectra can be directly correlated to the BF images, hence increased roughness observed in 500\,rpm films translate to a lowering of the overall reflectivity as a result of scattering. In the same way, similarities in the morphologies of films made above 1000\,rpm [Figs.\,\ref{4Fig1}(g-i)] lead to almost identical optical spectra. As C$_{12}$PI is a multilayer system, more excitons are available for absorption as the film thickness increases, thus the amplitude of the exciton dip in transmission spectra can be used as a gauge of the film thickness. From Fig.\,\ref{4Fig2}(b) we see that the film thickness decreases with spin speed as expected.

\begin{figure}[h!]
\centering
\includegraphics[width=\textwidth]{Fig3}
\caption{(a) Reflection and (b) transmission spectra for C$_{12}$PI films spin coated on silica at 4000\,rpm.}
\label{4Fig3}
\end{figure}
\subsection{Substrate preparation}
\label{sec:4-3}
Optical spectra of 4000\,rpm C$_{12}$PI films made using a variety of substrate preparation techniques are shown in Fig.\,\ref{4Fig3}. The reflectivity spectra are almost identical for all substrate preparations [Fig.\,\ref{4Fig3}(a)], with the exception of the silanised substrate where excess APTES molecules led to increased surface roughness, thus favouring the more crumpled and higher energy C$_{12}$PI phase. Removal of the excess silane via snowjetting creates flatter inorganic sheets and lowers the exciton energy. Dewetting of C$_{12}$PI films on non-functionalised substrates produces an increase in the film transmittance away from the exciton resonance [Fig.\,\ref{4Fig3}(b)].

\begin{figure}[h!]
\centering
\includegraphics[width=0.65\textwidth]{Fig4}
\caption{Degradation of 2000\,rpm C$_{12}$PI thin films shown in $100\times$ magnification BF images. Images of the sample were taken as-made (left), and after one week in standard conditions (right).}
\label{4Fig4}
\end{figure}
\subsection{Sample degradation}
BF images at 100$\times$ magnification of 2000\,rpm C$_{12}$PI films as-made (left) and after one week in standard conditions (right) are shown in Fig.\,\ref{4Fig4}. All films show signs of dewetting or diffusion, highlighting the importance of placing C$_n$PI films in low humidity atmospheres, or capping with polymer layers to prevent sample degradation \cite{Pradeesh2009}. The images from Fig.\,\ref{4Fig1} and spectra from Secs.\,\ref{sec:4-2} and \ref{sec:4-3} are recorded within one day of film production.

\begin{figure}[h!] 
\centering    
\includegraphics[width=0.9\textwidth]{Fig5}
\caption{BF images at 100$\times$ magnification of spin coated CHPI films on silica, with spin speed and substrate preparation as labelled. The black marks seen on (g-r) are due to dust particles on the microscope lens.}
\label{4Fig5}
\end{figure}
\section{CHPI thin films}
BF images of CHPI films at 100$\times$ magnification are shown in Fig.\,\ref{4Fig5}. No dewetting is observed with CHPI due to increased hydrophilicity of the organic molecule. Both snowjetting and high spin speeds improved the uniformity of samples [Figs.\,\ref{4Fig5}(a-f)], however the best films are produced with silanised substrates, regardless of spin speed [Figs.\,\ref{4Fig5}(g-l)]. A similar effect is seen for plasma etched substrates [Figs.\,\ref{4Fig5}(m-r)].

\subsection{Spin speed}
\label{sec:4-5}
\begin{figure}[h!] 
\centering    
\includegraphics[width=\textwidth]{Fig6}
\caption{Reflection and transmission spectra for CHPI films prepared on (a,b) untreated and (c,d) silanised silica substrates. (e) Effect of spin speed on CHPI film thickness on untreated substrates for 30\,mg/ml solutions. Error bars provide the standard deviation from 10 different films, and the dashed line represents a fit to $a\omega^{-b}$, with $b=0.45\pm0.01$.}
\label{4Fig6}
\end{figure}
Optical spectra of CHPI films created on untreated or silanised substrates are shown in Fig.\,\ref{4Fig6}, with the exciton resonance at the expected wavelength of 506\,nm \cite{Pradeesh2009b}. For untreated substrates, noticeable differences in the spectra between 2000 and 4000\,rpm films are caused by a morphology change: at low spin speeds increased film roughness produces lower overall reflectivity, and the appearance of a higher energy exciton leads to an apparent increase in the linewidth of the transmission dip [Figs.\,\ref{4Fig6}(a,b)]. Extra features that broaden the resonance peak have been observed in thick perovskite films ($>120$\,nm) and are attributed to stacking faults, strain and structural misalignment in the structure \cite{VijayaPrakash2009}. In contrast, as expected from their BF images [Fig.\,\ref{4Fig5}(g-i)] the spectra for silanised substrates exhibit the same features at all spin speeds [Figs.\,\ref{4Fig6}(c,d)], the main difference being a change in the amplitude of the exciton resonance as a result of the film thickness. 

Film thickness measurements for around 10 films on untreated substrates are fitted to $a\omega^{-b}$ with $b=0.45\pm0.01$ [Fig.\,\ref{4Fig6}(e)], close to the $\omega^{-0.5}$ relationship predicted by theory. Although this discrepancy may be attributed to the somewhat simplified model, a larger source of error comes from AFM measurements of film thickness. Film scratches are made using razor blades, thus AFM measurements may differ the true film thickness. It can also be difficult to provide accurate values due to film/substrate roughness, or small gradients in the film thickness. These effects are hard to quantify and not represented in Fig.\,\ref{4Fig6}(e), where the error bars are calculated from statistical analysis of the data recorded.

\begin{figure}[] 
\centering    
\includegraphics[width=\textwidth]{Fig7}
\caption{(a) Reflection and (b) transmission spectra (collected over areas with diameter $\approx 20\,$\textmu m) of 2000\,rpm CHPI films. Reflection and transmission spectra (diameter $\approx1\,$\textmu m) of 4000\,rpm CHPI films prepared on (c) untreated and (d) silanised and snowjetted silica substrates at the positions indicated on the images above.}
\label{4Fig7}
\end{figure}
\subsection{Substrate preparation}
Optical spectra of 2000\,rpm CHPI films made using a variety of substrate preparation techniques are shown in Fig.\,\ref{4Fig7}. The appearance of a second exciton due to structural misalignment [Sec.\,\ref{sec:4-5}] is seen for both untreated and snowjetted films. As indicated by their BF images [Fig.\,\ref{4Fig5}(h,k,n,q)], the spectra and morphologies of films made using other substrate preparations are very similar, and the uniformity is greatly improved by functionalisation of the substrate via silanisation or plasma etching. As an indication of the sample uniformity, line scans were made on 4000\,rpm films made using untreated [Fig.\,\ref{4Fig7}(c)], and silanised and snowjetted [Fig.\,\ref{4Fig7}(d)] substrates (signals collected over areas with diameter $\approx1\,$\textmu m). Cracks and discolourations can be seen in the case of the untreated substrate as a result of substrate non-uniformity [Fig.\,\ref{4Fig7}(c)], while the only defect seen for the silanised substrate comes from a piece of dust of the substrate [position 5 on Fig.\,\ref{4Fig7}(d)]. The near-identical spectra of all areas on the functionalised substrate is further proof of the film uniformity observed in BF images [Fig.\,\ref{4Fig7}(d)].

\begin{figure}[h!] 
\centering    
\includegraphics[width=\textwidth]{Fig8}
\caption{BF images at 100$\times$ magnification for 4000\,rpm CHPI films made on untreated substrates in (a) low humidity, (b) high humidity and (c) dehydrated spin coater atmospheres. (d) AFM measurements of above films.}
\label{4Fig8}
\end{figure}
\subsection{Humidity}
Hydrogen bonding between the organic and inorganic constituents is essential to assembly of the perovskite structure, therefore unwanted bonding or screening due to water molecules in the atmosphere can disrupt this process. A continuous CHPI film is formed at 4000\,rpm on an untreated substrate in a low humidity atmosphere [Fig.\,\ref{4Fig8}(a)], however the same spin coating conditions lead to dewetting at high humidity despite the hydrophilic organic group [Fig.\,\ref{4Fig8}(b)]. Ideally the spin coater should be desiccated as much as possible, and in order to achieve this the dehydration agent \ce{CaCl2} is placed inside the spin coater roughly one hour before film production. The spin coater is also pumped with \ce{N2} gas just before spinning, and the resulting film is very uniform even without the use of substrate functionalisation [Fig.\,\ref{4Fig8}(c)]. AFM measurements show that film roughness is reduced by a decrease in humidity as expected from BF images [Fig.\,\ref{4Fig8}(d)]. Films made in high humidity atmospheres have roughness on the order of the film thickness due to dewetting, while films made with CaCl have roughness $\sim5$\,nm.


\section{Conclusions}
Thin films of PbI perovskites with thickness $30-150$\,nm can be produced reliably using spin coating. Film morphology depends strongly on the organic molecule used in the perovskite, and dewetted films are produced for hydrophobic moieties. However film coverage and uniformity can be improved by using higher spin speeds, or substrate functionalisation techniques such as silanisation. Film thickness is controlled by the spin speed and initial solution concentration, and follows an $\omega^{-0.45}$ dependence, close to theoretical predictions. Formation of the perovskite structure can be disrupted by water in the atmosphere, and a dehydration agent should be placed in the spin coater to controllably produce a low humidity environment. The simplicity and adaptability of spin coating allows PbI perovskite thin films to be deposited on suitable substrates in order to create hybrid nanostructures.